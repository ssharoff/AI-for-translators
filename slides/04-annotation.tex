% for slides
%% \documentclass{beamer}
% for handout
\documentclass[handout]{beamer}

\usepackage{stdpresent}
\usepackage{latexsym}
\usepackage{graphicx}
\usepackage{url}
\renewcommand{\code}[1]{{\small \texttt{\red{#1}}}}
%% \newcommand{\G}[1]{{\selectlanguage{german}#1}}
%% \newcommand{\R}[1]{{\selectlanguage{russian}#1}}

%\usepackage{times}

\title{Linguistic annotation}
\subtitle{Mark up, XML, DTD, TEI}

\begin{document}

\maketitlepage

\section{Principles of annotation}
\subsection{What is annotation?}

\frame[<+->]{ \frametitle{}
 \includegraphics[width=\textwidth]{jane-eyre} 
}

\frame[containsverbatim]{
  \frametitle{Why annotation?}
 
{\scriptsize
      \begin{verbatim}
                                           JANE EYRE 479
                  CHAPTER XXXVII
                    CONCLUSION.
READER, I married him. A quiet wedding we had: he and 
I, the parson and clerk, were alone present. When we got 
back from church, I went into the kitchen of the manor-
house, where Mary was cooking the dinner, and John clean-
ing the knives, and I said : 
   ``Mary, I have been married to Mr. Rochester this morn-
ing.'' The housekeeper and her husband were both of that
      \end{verbatim}
}
\begin{itemize}
\item [] Any problems with OCR?
%\item [] page numbers, running titles, hyphenation, quotations, long dash, paragraph and sentence boundaries
\end{itemize}

}

\frame[<+->]{
  \frametitle{What is annotation?}

  \begin{itemize}
    \item Data vs. interpretation
    \item Markup contributes towards explicit interpretation of a text
    \item Early examples of markup: alphabet, punctuation, typographic setup
    \item Different annotations for different interpretations:\\
Chomsky set out his theory in \textit{Syntactic Structures} (1957)
    \item [$\to$]
\texttt{\tiny
Chomsky set out his theory in <i>Syntactic Structures</i> (1957)\\
Chomsky set out his theory in~<cite type="title$"$ ref="Chomsky1957"/>
}
  \end{itemize}
}

\subsection{XML: eXtensible Markup Language}

\frame[containsverbatim]{
  \frametitle{XML: example of code}
 
{\scriptsize

      \begin{verbatim}
<pb n="479"/> 
<div1 type="chapter"   n="38"> 
<h>Conclusion .</h>
<p><s>Reader , I married him .</s> 
<s>A quiet wedding we had : he and I , the parson and clerk , 
were alone present . </s> 
<s>When we got back from church , I went into the kitchen of 
the manor-house , where Mary was cooking the dinner , and John 
cleaning the knives, and I said : </s></p> 
<p><q><s> Mary, I have been married to Mr. Rochester this morning . </s> </q> 
...
</p>
      \end{verbatim}
}
}

\frame[<+->]{
  \frametitle{Markup languages}
  \begin{itemize}
    \item XML: eXtensible Markup Language (1999)\\
      Representing the logical structure of data
    \item XSL(T)= style sheet and transformation language
    \item HTML: HyperText Markup Language\\a relative of XML, xHTML
    \item TMX: Translation Memory eXchange format
    \item TBX: TermBase eXchange format
    \item SRX: Segmentation Rules eXchange format
    \item XML for dictionary entries
  \end{itemize}
}

\frame[<+->]{
  \frametitle{XML basics}
  \begin{itemize}
    \item tags, words in angular brackets: \code{<p> <div1>}
    \item elements, the content within a tag
    \item attributes and values: \code{<div1 type="chapter$"$ n="38$"$>}
    \item empty elements: \code{<pb n="474"/> }
    \item entities: \code{\&mdash;} (---), \code{\&amp;} (\&), \code{\&auml;} (\"a)
    \item DTD (Document Type Definition) --- \\definition of all the elements, attributes and entities
  \end{itemize}
}

\frame[<+->]{
  \frametitle{Well-formed XML document}
  \begin{itemize}
    \item every tag which is opened has to be closed
    \item [*] \code{<p> text <p> new paragraph}
    \item The only exception is for empty elements:\\ \code{<pb n="474"/> } $\mapsto$ \code{<pb n="474"></pb> }
    \item tags are properly embedded:
    \item [*] \code{<p> <q> quote </p></q>}
    \item Tags are case-sensitive:\\ \code{<publicationStmt> <gramGrp> }
  \end{itemize}
}

%% \frame{
%%   \frametitle{A dictionary entry}
%% {\scriptsize
%% \textbf{frustrate}
%% \begin{enumerate}
%%   \item VB:  frustrate   frustrates   frustrating   frustrated\\
%%   If something \textbf{frustrates} you, it upsets or angers you because you are unable to do anything about the problems it creates.\\
%%   \textit{These questions frustrated me.}\\
%%   \textit{Doesn't it frustrate you that audiences in the theatre are so restricted?}\\
%%    N-VAR  frustration   frustrations  \\
%%   \textit{The results show the level of frustration among hospital doctors.
%%   ...a man fed up with the frustrations of everyday life.}
%% \item   VB: frustrate   frustrates   frustrating   frustrated  
%%   If someone or something \textbf{frustrates} a plan or attempt to do something, they prevent it from succeeding.\\
%%   \textit{The government has deliberately frustrated his efforts to gain work permits for his foreign staff.\\
%%   ...her frustrated attempt to become governor.}
%% \end{enumerate}
%% }
%% }

\section{Linguistic annotation}

%% \frame[<+->]{
%%   \frametitle{Seven maxims (Leech, 1993)}
%%   \begin{enumerate}
%%     \item Possibility to remove annotations:
%%     \item [] Clair\_NP collects\_VVZ shoes\_NNS vs. 
%%     \\\code{<w pos="VVZ">collects</w>} vs.
%%     \\g/oing
%%     \item Possibility to extract annotations
%%     \item Guidelines available to the corpus user
%%     \item Clear methodology of annotation
%%     \item Annotation is interpretation
%%     \item Theory-neutral principles
%%     \item No annotation scheme has the right to be a standard vs. interoperability

%%   \end{enumerate}
%% }

\subsection{Formats of annotation}

\frame[containsverbatim]{
  \frametitle{Formats of annotation}
  \begin{itemize}
    \item Tab-separated format\\
{\scriptsize
      \begin{verbatim}
	N12:0510    VVD    studied    study    [Coord.N12:0504]
      \end{verbatim}
}
    \item COCOA: \code{<w VVD>studied}
    \item XML:
{\scriptsize
      \begin{verbatim}
	<s n="12">
        ...
	<w id="N12:0510">studied
	<ana lemma="study" pos="VVD" />
	</w>
	</s>
      \end{verbatim}
}
  \end{itemize}
}


\subsection{Types of annotation}

\frame[<+->]{
  \frametitle{Types of annotation}
  \begin{itemize}
    \item Metatext annotation (author, audience, domain, genre)
    \item Text annotation (quotes, comments, page breaks)
    \item Typographic annotation (fonts, headings, text alignment)
    \item Linguistic annotation:
    \item [] Part-of-speech (POS): \code{<w pos="VVZ$"$>studies</w>}
    \item [] Lemmatisation: \code{He <w lemma="leave$"$>left</w>}
%%     \item [] Syntactic structure: Suzanne or PennTreeBank\\
%% \code{\scriptsize
%% 	N12:0510    VVDv    studied    study    [Vd.Vd]
%% }
    \item [] Functional relationships, e.g., coreference:\\\code{<w ref="X1" lemma="he$"$>He</w>}
    \item [] Word senses: power$_1\to$énergie, power$_2\to$pouvoir
      \setlength{\tabcolsep}{2pt}
    \item [] \begin{tabular}{lllllll}
The owner & of & the company & has & the power & to fire & any workers\\
Le propriétaire & de & la société & a & le pouvoir & de licencier & les travailleurs
      \end{tabular}
  \end{itemize}
}

\frame[<+->]{ \frametitle{Other kinds of annotation}
\begin{itemize}
  \item Named Entities\\
Mark boundaries of names of type PERson, ORGanization, GPE, LOCation,\ldots
\item [] \code{<enamex type="ORGANIZATION$"$>}FBI\code{</enamex>} agents arrested \code{<enamex type="PERSON$"$>}Kaczynski\code{</enamex>} on \code{<enamex type="DATE$"$>}April 3, 1996\code{</enamex>}, at his remote cabin outside \code{<enamex type="LOCATION$"$>}Lincoln\code{</enamex>}
\item Gazetteers, normalisers, disambiguation
\item []
  \textit{Lech~Kaczynski} vs  \textit{Jaroslaw Kaczynski} vs \textit{Ted~Kaczynski}
  \textit{Lincoln, US} vs  \textit{Lincoln, UK} 
\item Terminology annotation: Identify terms, detect their canonical form and link to their database records
\item []
  \code{<te id=$"$1123$"$>}Fast Breeder Reactors\code{</te>} (\code{<te id=$"$1123$"$ type=$"$abbr$"$>}FBR\code{</te>}) can produce more \code{<te id=$"$1134$"$>} fissile fuel\code{</te>} than \code{<te id=$"$1123$"$ type=$"$ana$"$>} they\code{</te>} consume.
\end{itemize}
}

\frame[<+->]{
  \frametitle{Standards of annotation}
  \begin{itemize}
    \item TEI (Text Encoding Initiative), TEI-Lite (a subset) --- for text and metatext annotation
    \item EAGLES (European Advisory Group on Language Engineering Standards) --- for linguistic annotation
    \item \textit{The man still saw her}
    \item [] The ($AT$) man ($NN|VV$) still ($NN|VV|RB$) saw ($NN|VVD$) her ($PP|PP\$$) . ($SENT|PUN$)
    \item Rules vs statistical training
    \item Statistical hidden Markov model: $p(t_i|w_i)\times p(t_i|t_{i-1})$
    \item [] \begin{tabular}{lll}
$p(PP|her)=0.3;$ & $p(PP|SENT)=0.0022;$ & 	=0.000654\\
$p(PP\$|her)=0.7;$ & $p(PP\$|SENT)=0.000019$ &	=0.000013
      \end{tabular}
  \end{itemize}
}

\frame[<+->]{
  \frametitle{POS tags}
  \begin{itemize}
    \item The standard inventory for English (the Penn set):\\
	VV -- base verb, VVD -- past tense, VVN -- pp (taken)\\
	NN -- common noun, singular; NNS -- common noun, plural;\\
	CS -- conjunction, subordinative, \textit{so that}: so\_CS21 that\_CS22
      \item [] Penn set: 55 tags; Lancaster set: 146 tags
      \item Specific sets for languages and taggers:\\
	German STTS: VVFIN -- finite verb, VVINF -- infinitive,
	VVIZU -- infinitive with ``zu'', \\NN -- common noun (50 tags)
      \item [] Japanese mecab tagset: 16 tags 
      \item [] Russian tagset: 1066 tags
      \item [] $Ncmsan:$ Noun, Type = common, Gender = masculine,
Number = singular, Case = accusative, Animate = no

  \end{itemize}
}

\frame[<+->]{
  \frametitle{Accuracy of tagging}
  \begin{itemize}
    \item Accuracy -- the percentage of words (i.e. word tokens) which are correctly tagged
    %% \item Recall is the extent to which all correct annotations are found in the output of the tagger.
    %% \item Precision is the extent to which incorrect annotations are rejected from the output.
    \item [] 95\% accuracy: one in 20 is wrong
    \item [] 96\% accuracy: one in 25 is wrong
    \item [] an improvement of 25\% from 95\% to 96\% ???
    \item Domain and genre influence:\\
      newspapers for training (Wall Street Journal in Penn Treebank)
  \end{itemize}
}

\frame[<+->]{ \frametitle{Types of errors}
\begin{itemize}
  \item \textit{I can light a fire. You can open a can.}
  \item [$\to$] PP VV NN DT NN. PP VV VV DT VV.
  \item [] \includegraphics[width=0.7\textwidth]{precision-recall} 
  \item Precision = TP/(TP+ FP) vs \\Recall = TP/(TP + FN)
%% https://towardsdatascience.com/model-evaluation-i-precision-and-recall-166ddb257c7b
\end{itemize}
}

\frame[<+->]{ \frametitle{Text level annotation}
\begin{itemize}
  \item Linguistic features which can be counted
  \item [] Lexical features:\\
$publicVerbs = acknowledge, admit, agree, assert, claim...$\\
%% $timeAdverbials = afterwards, again, earlier, early, ...$\\
$amplifiers = absolutely, altogether, completely, enormously,...$
  \item []
    Part-of-speech features:\\
Nominalisations (nouns ending in $-tion, -ness, -ment$)\\
main POS tags\\
Past tense verbs (VVD)
\item []
  Syntactic features:\\
$that$ deletions\\
pied piping (\textit{Which house did she buy ...?})
\item []
  Text-level features, such as:\\
Average word length\\
Average sentence length
  \item Factor analysis: which features group with each other
  %% \item [$\to$] Text dimensions: informational, narrativity, situation-dependent, etc
\end{itemize}
}
\frame[<+->]{ \frametitle{Multi-dimensional analysis (Biber, 1988)}
 \includegraphics[height=0.8\textheight, width=0.8\textwidth]{biber-dim1-3} 
}
\frame[<+->]{ \frametitle{Multi-dimensional analysis (Biber, 1988)}
  \begin{tabular}{lll}
Functions 	& Linguistic features 	& Characteristic genres\\\hline\hline
Dimension 1\\\hline
Monologue 	& nouns, adjectives	& informational exposition\\
Careful production & prepositional phrases &e.g., official documents\\
Faceless	& long words		& academic prose\\\hline
Interactive	& 1$^{st}$ and 2$^{nd}$ PPs 	& conversations\\
Personal focus	& questions, reductions & (public and private)\\
Involved	& stance verbs, hedges\\
Online production& emphatics\\\hline\hline
\pause
Dimension 3\\\hline
Elaborated	& wh-relative clauses 	& official documents\\
		& pied-piping  		& professional letters\\
		& phrasal coordination 	& (exposition) \\\hline
Situation-  	& time and place 	& broadcasts\\
dependent 	& \hspace{2em} adverbials & (fiction)\\

  \end{tabular}
}

%% \frame[containsverbatim,<+->]{
%%   \frametitle{Annotations for syntactic parsing}
 
%% {\scriptsize
%%       \begin{verbatim}
%% <cl type='finite declarative' function='independent'>
%%   <phr type='NP' function='subject'>Nineteen fifty-four,
%%     <cl type='finite relative declarative' function='appositive'>when
%%       <phr type='NP' function='subject'>I</phr>
%%       <phr type='VP' function='predicate'>was eighteen
%%                     years old</phr>
%%     </cl>,
%%   </phr>...
%% </cl>
%%       \end{verbatim}
%% }

%% }

%% \frame[containsverbatim,<+->]{
%%   \frametitle{Representing alignment}
 
%% {\scriptsize
%% %      \begin{verbatim}
%%   \begin{block}{A sentence from \textit{Alice in Wonderland}}
%%     \begin{itemize}
%%       \item [] There was nothing so VERY remarkable in that; nor did Alice think it so VERY much out of the way to hear the Rabbit say to itself `Oh dear! Oh dear! I shall be late! 
%%       \item [] Dies war grade nicht sehr merkwürdig; Alice fand es auch nicht sehr außerordentlich, daß sie das Kaninchen sagen hörte: ``O weh, o weh! Ich werde zu spät kommen!"
%%       \item [] \R{    Конечно, ничего удивительного в этом не было. Правда, Кролик на бегу говорил: --- Ах, боже мой! Я опаздываю. Но и это не показалось Алисе особенно странным.} 
%%     \end{itemize}
%%   \end{block}
%% %      \end{verbatim}
%% }
%% \begin{itemize}
%% \item Text / paragraph / sentence / word-level alignment
%% \item Existing parallel corpora:\\
%% Europarl, Hansards, Open Office
%% \end{itemize}

%% }

\frame{
  \frametitle{Basic points}
  \begin{itemize}
    \item Raw texts vs. annotated texts
    %% \item Leech's maxims for text annotation
    \item Annotation formats and standards: XML, TEI, EAGLES
    \item Metatextual annotation
    \item POS tagging
    \item Higher-level annotation
      \begin{block}{For the next seminar}
	Prepare for a short presentation of the most exciting findings in your projects to discuss them with another student in the group
      \end{block}
  \end{itemize}
}

\end{document}
