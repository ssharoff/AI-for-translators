% for slides
\documentclass{beamer}
% for handout
%% \documentclass[handout]{beamer}

\usepackage{stdpresent}
%\newcommand{\R}[1]{{\selectlanguage{russian}#1}}

%\usepackage{times}

\title{Statistics of collocations}

\begin{document}

\maketitlepage

\section{Collocations}
\subsection{Definitions}
\frame[<+->]{
  \frametitle{Collocations}

  \begin{itemize}
    \item `Collocations of a given word are statements of the habitual or customary places of that word' (Firth, 1957) \\`you shall know a word by the company it keeps'
    \item collocations -- constructions; collocates -- words in collocations, 
    \item Non-compositionality: \textit{strong} vs. \textit{powerful}: \textit{tea} or \textit{car}
      \\ \textit{released from prison} vs \textit{discharged from hospital}
      \\ \textit{full knee replacement} vs \textit{total knee replacement}
    \end{itemize}
  \begin{block}{Examples of collocations}
    \begin{itemize}
    \item Terms: \textit{stiff breeze, weapons of mass destruction}
    \item Phrasal verbs: \textit{get off, tell off, look up, make up}
    \item Support verb constuctions: \textit{take a shower, make sense, }
    \item Stock phrases: \textit{the rich and powerful, by and large}
    \end{itemize}
  \end{block}
}

\subsection{Methods for counting}
\frame[<+->]{ \frametitle{Methods for counting}
\begin{itemize}
   \item N-grams: sequences of N words (bi-, trigram)
   \item \textit{to be or not to be}\\
     unigrams $\to$ \textit{to, be} (2), \textit{or, not} (1)\\
     bigrams $\to$ \textit{to be} (2);  \textit{be or} (1);  \textit{or not} (1);  \textit{not to} (1)\\
     trigrams $\to$ \textit{to be or} (1);  \textit{be or not} (1);  \textit{or not to} (1)
   \item Skip-grams: pairs in a window of N words:\\
     W2: \textit{to be} (2);  \textit{to or};  \textit{be or};  \textit{be not}
\end{itemize}
}

\frame[<+->]{ \frametitle{Counting bigrams}
  \begin{itemize}
  \item [] 
{\footnotesize
\begin{tabular}{|l|r||l|r|}
\hline
\multicolumn{2}{|c||}{\textbf{Bigrams}}&\multicolumn{2}{c|}{\textbf{Trigrams}}\\
\hline
 of the&   7211.67&i       do      not &522.24\\
 in the&   5167.19&there   be      a&	401.55\\
 it be&	   4050.64&it      be      a&	372.39\\
 to the&   2617.17&one     of      the&	356.03\\
 be a&	   2366.99&it      be      not&	348.88\\
 do not&   2230.41&there   be      no&	292.65\\
 on the&   2181.97&be      able    to&	241.46\\
 have be&  2151.05&do      not     know&232.90\\
 there be& 2017.23&the     end     of&	213.57\\
\hline
  \end{tabular}
}

  \item []
{\footnotesize
 \begin{tabular}{|l|r|}
\hline
  last    year&      107.22 \\
  prime   minister&  97.18  \\
  last    night&     84.95  \\
  first   time&	     83.27  \\
  other   hand&	     56.12  \\
  last    week&	     51.27  \\
  other   people&    42.01  \\
  next    year&	     40.35  \\
  soviet  union&     38.95  \\
  young   man&	     38.29  \\
\hline
  \end{tabular}
}
  \end{itemize}
}

\section{Statistical measures}
\subsection{Notions of probability}
\frame[<+->]{
  \frametitle{Statistics of surprise}
  \begin{itemize}
     \item Null hypothesis: words are distributed at random
     \item $F_i$ --- number of occurrences of word$_i$
     \item $F_{ij}$ --- number of joint occurrences of the two words (i and j)
     \item $N$ --- corpus size
     \item $O_{ij}$ --- observed probability, $E_{ij}$ --- expected probability, 
     \item $O_{ij}=\frac{F_{ij}}{N}$ --- observed probability, 
     \item $E_{ij}=\frac{F_i}{N}\times\frac{F_j}{N}$ --- expected probability, 
\item []
\footnotesize{
  \begin{tabular}{|c| c c c c c c|}
\hline
    &1&2&3&4&5&6\\
\hline
   1&.&.&.&\textbf{x}&.&.\\
   2&.&.&.&.&.&.\\
   3&.&.&.&.&.&.\\
   4&.&.&.&.&.&.\\
   5&.&.&.&.&.&.\\
   6&.&.&.&.&.&.\\
\hline
  \end{tabular}
}
  \end{itemize}
}

\frame[<+->]{ \frametitle{Notation for the probabilities}
\begin{itemize}
  \item The space of events (S)\\
    What is our event space?
  \item $p(x | \text{partial knowledge})$
  \item Conditional independence:\\
    Knowing about X doesn't tell me about Y
  \item [] $p(Y | X ) = p( Y )$\\
    $p(X | Y ) = p( X )$
  %% \item NOT in language
  \item Conditional probability\\
    \vspace{1ex} $p(X | Y ) = \frac{p(X \& Y)}{p(Y)}$
  \item  \includegraphics[width=0.6\textwidth]{align3} \\
$p(house|maison)=0.476$\\
$p(home|maison)=0.104$\\
$p(parent|maison)=0.077$\\
$p(flower|fleur)=0.039$\\
$p(pollen|fleur)=0.020$
\end{itemize}
}



\subsection{Statistics of surprise}
\frame{
  \frametitle{Measures of collocations}
  \begin{itemize}
     \item $O_{ij}=\frac{F_{ij}}{N}$ --- observed probability, 
     \item $E_{ij}=\frac{F_i}{N}\times\frac{F_j}{N}$ --- expected probability, 
     \item $MI_{ij}=\log \left( {\frac{O_{ij} }{E_{ij} }} \right)$ --- Mutual Information score,
     \item $Dice_{ij}={2 \times \frac{O_{ij} }{E_i+E_j }} $ --- Dice score,
     \item $T_{ij}=\frac{O_{ij} -E_{ij} }{\sqrt {O_{ij} } }$ --- T-score
     \item Log-likelihood (LL) score from contingency table\\
       \begin{tabular}{l l l}
            & word2    & $\neg$ word2 \\\hline
       word1& $F_{ij}$ & $F_i-F_{ij}$ \\\hline
$\neg$ word1& $F_j-F_{ij}$ & $N-F_{ij}$ \\\hline
       \end{tabular}
     \item [] $loglike=2(a\ln(\frac{F_i}{E1})+b\ln(\frac{F_j}{E2})); E1=c\frac{a+b}{c+d}; E2=d\frac{a+b}{c+d}$
  \end{itemize}
}

\frame{ \frametitle{Examples of predictions}
\begin{itemize}
     \item \textit{new company}, $F_{ij}=358, F_i=105,645, F_j=57,118, N=100,000,000$
     \item \textit{private company}, $F_{ij}=423, F_i=16,357, F_j=57,118, N=100,000,000$
     \item \textit{post office}, $F_{ij}=1,425, F_i=10,871, F_j=29,132, N=100,000,000$
%% 	 $LL_{ij}=$&$\log L(a,c_1,p) + \log L (b, N-c_1, p) - $\\
%% 	 &$\log L(a,c_1,p_1) - \log L(c, N-c_1, p_2)$\\
%% \\
%% 	 &$L(k,n,x)=x^k(1-x)^{n-k}$
\end{itemize}
       \begin{tabular}{lrrrr}
                & MI score & Dice & T-score & LL-score \\
\textit{new company}	&	6.19 &	2.82 &	15.97&	761.32\\
\textit{private company}&	5.74 &	7.61 &	20.18&	2,548.55\\
\textit{post office}	&	8.59 &	9.44 &	25.11&	6,354.51\\
       \end{tabular}
}

%% \section{Implications of frequency analysis}
%% \subsection{Logarithms and Zipf's law}
%% \frame[<+->]{
%%   \frametitle{Logarithms and Zipf's law}
%%   \begin{itemize}
%%      \item [] 
%%        \begin{tabular}{l l}
%% 	$\log_2(4)=2$	&$\leftarrow 2^2=4$\\
%% 	$\log_2(16)=4$	&$\leftarrow 2^4=16$\\
%% 	$\log_2(256)=8$	&$\leftarrow 2^8=256$
%%        \end{tabular}
%%      \item [] $\log_2(1024)=10 \leftarrow 2^{10}=1024$
%%      \item $\log(a\times b)=\log(a)+\log(b)$
%% %     \item $\log_2(6187267)=10$
%%      \item [] : 
%%     \begin{tabular}{|r|l|r|}
%%       \hline
%%       Frequency	&Word form	&Rank\\
%%       \hline
%% 6187267	&the	&1\\
%% 2941444	&of	&2\\
%% 2682863	&and	&3\\
%% 2126369	&a	&4\\
%% 1812609	&in	&5\\
%% 47141	&member	&200\\
%% 10071	&useful	&1000\\
%% 1024	&miserable	&5000\\
%%       \hline
%%     \end{tabular}

%%   \end{itemize}
%% }

%% \frame[<+->]{
%%   \frametitle{Zipfian distribution in the BNC}
%%   \begin{itemize}
%% \item []
%%     \begin{tabular}{|r|l|r|r|r|}
%%       \hline
%%       Frequency	&Word	&Rank	&Log(frq)&Log(rank)\\
%%       \hline
%% 6187267	&the	&1&22.56&0.00\\
%% 2941444	&of	&2&21.49&1.00\\
%% 2682863	&and	&3&21.36&1.58\\
%% 2126369	&a	&4&21.02&2.00\\
%% 1812609	&in	&5&20.79&2.32\\
%% 47141	&member	&200&15.52&7.64\\
%% 10071	&useful	&1000&13.3&9.97\\
%% 1024	&miserable	&5000&10.00&12.29\\
%%       \hline
%%     \end{tabular}
%% % \item [] $\log(frq) \approx C-\log(rank)$
%% % 	$\log(frq1)=C-\log(1)$
%%   \end{itemize}
%% }

%% \frame[<+->]{
%%   \frametitle{Frequency to rank ratio}
%% \pgfimage[width=\textwidth]{zipf-l}
%% }

%% \frame[<+->]{
%%   \frametitle{Log frequency to log rank ratio}
%% \pgfimage[width=\textwidth]{zipf-log}
%% }

\section{Implications of collocations}
\subsection{Collocations in a window}
\frame[<+->]{
  \frametitle{Corpus analysis}
				
  \begin{itemize}
    \item Multiword terminology\\
      Multiterm Extract
    \item One sense per collocation hypothesis\\
      \textit{take kindly}
    \item Queries for collocations:\\
      \textit{strong} N.*
    \item [] Right window of 3: \textit{to offer} N.*
    \item Collocations for other languages\\
      \textit{\textbf{den Vorteil} eines persönlichen Kontaktes über die Stimme	\textbf{bietet}.}\\
      \textit{offer the advantage}
  \end{itemize}
}

%contingency tables for parallel texts


\subsection{Word sketches}
\frame{ \frametitle{Automation through word sketches}
\begin{itemize}
    \item Word sketches in \url{http://the.sketchengine.co.uk/}
    \item Fixed set of queries for Intellitext:\\
      Modifiers: ADV .. V.*\\
      Objects: V.* .. N.* or N.* \textit{to be} VVN
    \item Sketches for other languages\\
      \textit{bieten}
\end{itemize}
}

\frame{
  \frametitle{Basic points}
				
  \begin{itemize}
    \item Collocations and collocates
    \item Statistics for measuring surprise
    \item Human judgment vs. computer model
      \begin{block}{For the seminar}
	Study collocation properties for words in your projects\\
	Use their immediate left/right contexts and spans;\\
	Try filtering collocates by their POS tags\\
        Use word sketches
      \end{block}
      \begin{block}{For further classes}
	Please either install Python and Jupyter Lab on your own laptop:\\
        \url{https://jupyter.org/install}\\
        or make sure you have access to Google Drive and Google Colab:\\
        \url{https://colab.research.google.com/}
      \end{block}
  \end{itemize}
}
\end{document}


